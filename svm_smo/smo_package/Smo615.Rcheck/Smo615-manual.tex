\nonstopmode{}
\documentclass[a4paper]{book}
\usepackage[times,inconsolata,hyper]{Rd}
\usepackage{makeidx}
\usepackage[utf8,latin1]{inputenc}
% \usepackage{graphicx} % @USE GRAPHICX@
\makeindex{}
\begin{document}
\chapter*{}
\begin{center}
{\textbf{\huge Package `Smo615'}}
\par\bigskip{\large \today}
\end{center}
\begin{description}
\raggedright{}
\item[Type]\AsIs{Package}
\item[Title]\AsIs{2D Visualization of SVM using the SMO algorithm}
\item[Version]\AsIs{1.0}
\item[Date]\AsIs{2018-12-13}
\item[Author]\AsIs{Rui Ma, Shaocheng Wu, Mingyu Du}
\item[Maintainer]\AsIs{Biostat 615 }\email{biostat615@umich.edu}\AsIs{}
\item[Description]\AsIs{We implement a support vector classifier using SMO algorithm in C++ and
compile it into this R package. It seeks to find a decision boundary with the greatest
margin of separation between the input classes. If the training data are not linearly
separable, a ``soft-margin'' (or regularized) version can be implemented that allows for
misclassified samples. The output decision boundary can be linear, as it is with
logisitic regression, or non-linear with the use of various kernels that map the input
features to a higher-dimensional space.}
\item[License]\AsIs{GPL (>= 2)}
\item[Imports]\AsIs{Rcpp (>= 0.12.16)}
\item[LinkingTo]\AsIs{Rcpp}
\end{description}
\Rdcontents{\R{} topics documented:}
\inputencoding{utf8}
\HeaderA{Smo615-package}{2D Visualization of SVM using the SMO algorithm}{Smo615.Rdash.package}
\aliasA{Smo615}{Smo615-package}{Smo615}
\keyword{SMO; SVM}{Smo615-package}
%
\begin{Description}\relax
We implement a support vector classifier using SMO algorithm in C++ and compile it into this R package. It seeks to find a decision boundary with the greatest margin of separation between the input classes. If the training data are not linearly separable, a "soft-margin" (or regularized) version can be implemented that allows for misclassified samples. The output decision boundary can be linear, as it is with logisitc regression, or non-linear with the use of various kernels that map the input features to a higher-dimensional space.
\end{Description}
%
\begin{Details}\relax
We define a linear kernel and a Gaussian (also known as radial basis function or RBF) kernel. There are three main steps to train our model: take\_step(), examine\_example(), and train(). The train() function implements selection of the first α to optimize via the first choice heuristic and passes this value to examine\_example(). Then examine\_example() implements the second choice heuristic to choose the second α
to optimize, and passes the index of both α values to take\_step(). Finally take\_step() carries out the meat of the calculations and computes the two new α values, a new threshold b, and updates the error cache. The train() function uses a while loop to iterate through the α values in a few different ways until no more optimizations can be made, at which point it returns the optimized α vector. With all the trained parameters, we conduct the predictions of the testing data and also plot the decision boundary for both the training data and testing data.
\end{Details}
%
\begin{Author}\relax
Rui Ma, Shaocheng Wu, Mingyu Du

Maintainer: Biostat 615 <@umich.edu>
\end{Author}
%
\begin{References}\relax
1.Platt, J.: Sequential Minimal Optimization: A Fast Algorithm for Training Support Vector Machine. Technical Report MSR-TR-98-14. Microsoft research (1998).
2.Jon Charest: https://jonchar.net/notebooks/SVM/.
3.Wikipedia contributors. Support Vector Machine. Wikipedia, The Free Encyclopedia. Available at:https://en.wikipedia.org/wiki/Support\_vector\_machine. Accessed December 1, 2018.
\end{References}
%
\begin{SeeAlso}\relax
Optional links to other man pages
\end{SeeAlso}
%
\begin{Examples}
\begin{ExampleCode}
  ## Not run: 
     ## Example
     library(Smo615)
     res=smo615(train_data_path, train_label_path, test_data_path, test_label_path, C_regularization, linear_kernel(0) /Gaussian_kernel(1))
  
## End(Not run)
\end{ExampleCode}
\end{Examples}
\printindex{}
\end{document}
